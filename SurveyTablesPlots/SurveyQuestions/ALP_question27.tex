\begin{table}[!h]
\rowcolors{1}{blue!30}{blue!10}
\scriptsize
%\footnotesize
\adjustbox{max height=\dimexpr\textheight-1cm\relax,max width=2\textwidth}
{
\setlength{\extrarowheight}{15pt}
 \begin{tabular}{|c|c|c|c|}\hline
\bf{Qu.\#} & \bf{Question ID} &\bf{Question Text}&\parbox[c][0.05\textheight][c]{0.18\textwidth}{\bf{Notes}}\\ \hline \hline 

27&  \parbox[c][0.05\textheight][c]{0.17\textwidth} {  perceivedunderreport-auditpenalty\_\{a,b,...,h\} }  &\parbox[c][0.17\textheight][c]{0.5\textwidth} {Let's consider how low the effective income tax rate would need to be before everyone reported 100\% of their taxes to the IRS, assuming there are no audits or penalties.  For this question, assume for the moment that everyone has the same effective tax rate.
If each of the effective income tax rates below were applied to everyone, please indicate if you think (a) the majority of people like you would report their full income OR (b) the majority of people like you would underreport their income: 
(a) Income tax rate = 1\% 
(b) Income tax rate = 2.5\%
(c) Income tax rate = 5\%
(d) Income tax rate = 10\%
(e) Income tax rate = 15\%
(f) Income tax rate = 20\%
(g) Income tax rate = 25\%
(h) Income tax rate = 30\%
}
&\parbox[c][0.15\textheight][c]{0.18\textwidth}{}\\  \hline
\end{tabular}
}
\end{table}