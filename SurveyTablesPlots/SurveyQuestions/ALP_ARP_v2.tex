
\begin{table}[!h]
\rowcolors{1}{blue!30}{blue!10}
%\scriptsize
\footnotesize
\adjustbox{max height=\dimexpr\textheight-1cm\relax,max width=2\textwidth}
{
\setlength{\extrarowheight}{5pt}
 \begin{tabular}{|c|c|c|}\hline
\bf{Qu.\#}& \bf{Question ID} & \bf{Question Text}\\ \hline \hline 

8&  \parbox[c][0.05\textheight][c]{0.17\textwidth} { perceivedauditrate \&  -magnifier }  & \parbox[c][0.05\textheight][c]{0.68\textwidth} {   In a typical year, what percent of taxpayers in the U.S. will have their income tax return audited by the IRS?
 }
\\  \hline

9a& \parbox[c][0.05\textheight][c]{0.17\textwidth} {  perceivedar-underreport  } & \parbox[c][0.08\textheight][c]{0.68\textwidth} {   Imagine a taxpayer that only paid [30\% of the taxes he or she owes, 60\% of the taxes he or she owes, 90\% of the taxes he or she owes]. Do you think the chances of that person being audited that year would be higher, lower, or the same as if he or she had paid all taxes owed? 
 }
\\  \hline

9b& \parbox[c][0.04\textheight][c]{0.17\textwidth} {  perceivedaruprob \&\\  -magnifier} & \parbox[c][0.04\textheight][c]{0.68\textwidth} {    What is the percent chance that person will have their income tax return audited by the IRS?
 }
\\  \hline

10a& \parbox[c][0.04\textheight][c]{0.17\textwidth} {  bombcrater } &  \parbox[c][0.05\textheight][c]{0.68\textwidth} {    If your tax return was audited last year, do you think the chances of being audited the following year are higher, lower, or the same?  }
\\  \hline


10b& \parbox[c][0.04\textheight][c]{0.17\textwidth} {  bombcrateramount \&  -magnifier  } &  \parbox[c][0.04\textheight][c]{0.68\textwidth} {    What do you think are these new chances of getting audited?
 }
\\  \hline

11&  perceivedpenaltyrate  &\parbox[c][0.15\textheight][c]{0.68\textwidth} {   Now let's consider the penalty rate. If the IRS detects that a person has underreported their taxes, they will first have to pay the unpaid taxes that were due. In addition, they will be assessed a penalty that is a percentage of the amount they underpaid. This percentage is the penalty rate. Imagine a person was caught underpaying their taxes by \$1000. In addition to having to pay that \$1000, how much of a penalty would they have to pay?\\
{\bf NOTE:} We converted this response to a peproportion so that in our analysis perceivedpenaltyrate  represents a proportion relative to \$1000. 
 }
\\  \hline

12 &   perceivedtaxrate  & \parbox[c][0.05\textheight][c]{0.68\textwidth} { What do you think your effective income tax rate was this past year?
Visual 0-100 subjective probability slider [with checks that response is between 0 and 100]
 }
\\  \hline

13& \parbox[c][0.05\textheight][c]{0.17\textwidth} {  perceivedevasion- ratepopulation }  &\parbox[c][0.04\textheight][c]{0.68\textwidth} {    In a typical year, out of all taxpayers in the United States, what percent intentionally underreport their taxes?
 }
\\  \hline

14&  \parbox[c][0.05\textheight][c]{0.17\textwidth} { perceivedevasionrate}  &\parbox[c][0.04\textheight][c]{0.68\textwidth} {    Now consider people like you. In a typical year, out of 100 people like you, how many intentionally underreport their taxes?
 }\\  \hline

15& \parbox[c][0.05\textheight][c]{0.17\textwidth} {  perceivedevasion-manyevaders}   &\parbox[c][0.06\textheight][c]{0.68\textwidth} {   Imagine that a widely-disseminated news story comes out that half of all US taxpayers underreport their taxes. Out of 100 people like you, how many would now underreport their taxes?
 }
\\  \hline

16& \parbox[c][0.03\textheight][c]{0.17\textwidth} {  perceivedcaught}    &\parbox[c][0.03\textheight][c]{0.68\textwidth} {    In a typical year, what percent will be caught by the IRS?
 }\\  \hline


\end{tabular}
}
\end{table}
